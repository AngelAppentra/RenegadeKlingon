% Título, autor, fecha. 
\title{capitulo5} 
\author{Adrián Brañas Castro}
\date{\Large Enero, 2013} 

%capitulo 5 Metodologia
\chapter {Metodología}
\label{capitulo5}

En este capítulo se va a explicar la metodología de desarrollo software que se ha empleado para la realización de este Proyecto de Fin de Carrera. En primer lugar, haremos una descripción de lo que es una metodología ágil de desarrollo (Sección \ref{capitulo5:metodologia}), a continuación explicaremos la metodología Scrum (Sección \ref{capitulo5:scrum}) y finalmente describiremos la dinámica de las reuniones de Scrum para desarrollo colaborativo (Sección \ref{capitulo5:reuniones}) .

\section{Metodología ágil de desarrollo}
\label{capitulo5:metodologia}
El desarrollo ágil de software consiste en métodos de ingeniería del software basados en el desarrollo iterativo e incremental, donde los requerimientos y soluciones evolucionan mediante la colaboración de grupos auto organizados y multidisciplinarios. Existen muchos métodos de desarrollo ágil; la mayoría minimiza riesgos desarrollando software en fases cortas. Una metodología ágil es, por ejemplo, la metodología Scrum, que se explicará en el siguiente apartado. El software desarrollado en una unidad de tiempo es llamado iteración, la cual debe durar de una a cuatro semanas; en nuestro caso dura exactamente 2 semanas laborales. Cada iteración del ciclo de vida incluye: planificación, análisis de requerimientos, diseño, codificación, revisión y documentación. Una iteración no debe agregar demasiada funcionalidad, pero la meta es tener un producto con más características y sin errores al final de cada una de ellas. 

Al final de cada iteración el equipo vuelve a evaluar las prioridades del proyecto.
Los métodos ágiles enfatizan las comunicaciones cara a cara en vez de la documentación. En las reuniones que se hacen para cada ciclo, deben estar presentes revisores, escritores de documentación y ayuda, diseñadores de cada iteración y directores de proyecto. Los métodos ágiles también enfatizan que el software funcional es la primera medida del progreso.

En la figura \ref{fig:agil} se representa las diferentes fases y ciclos de los que consta esta metodología.

%figura metodologia agil
\begin{figure}[t]
\begin{center}
\includegraphics[width=0.6\linewidth]{includes/images/agil.png}
\caption{Esquema de las diferentes fases de la metodología ágil}
\label{fig:agil}
\end{center}
\end{figure}



\section{Metodología Scrum}
\label{capitulo5:scrum}
Scrum es un marco de trabajo para la gestión y desarrollo de software basada en un proceso iterativo e incremental utilizado comúnmente en entornos basados en el desarrollo ágil de software [SCRUM 2012].
Aunque Scrum estaba enfocado a la gestión de procesos de desarrollo de software, puede ser utilizado en equipos de mantenimiento de software, o en una aproximación de gestión de programas.

Scrum es un modelo de referencia que define un conjunto de prácticas y roles, y que puede tomarse como punto de partida para definir el proceso de desarrollo que se ejecutará durante un proyecto. Los roles principales en Scrum son el ScrumMaster, que mantiene los procesos y trabaja de forma similar al director de proyecto, el ProductOwner, que representa a los stakeholders (interesados externos o internos), y el Team que incluye a los desarrolladores.
Durante cada sprint, un periodo entre una y cuatro semanas (la magnitud es definida por el equipo; en nuestro caso 2 semanas como ya habíamos mencionado), el equipo crea un incremento de software potencialmente entregable (utilizable). El conjunto de características que forma parte de cada sprint viene del Product Backlog, que es un conjunto de requisitos de alto nivel priorizados que definen el trabajo a realizar. Los elementos del Product Backlog que forman parte del sprint se determinan durante la reunión de Sprint Planning. Durante esta reunión, el Product Owner identifica los elementos del Product Backlog que quiere ver completados y los comenta a todo el equipo. A continuación, se determina la cantidad de ese trabajo que se podrá completar durante el siguiente sprint. Durante el sprint, nadie puede cambiar el Sprint Backlog, lo que significa que los requisitos están congelados durante dicho sprint.

Scrum permite la creación de equipos auto organizados impulsando la co-localización de todos los miembros del equipo, y la comunicación verbal entre todos ellos.
Un principio clave de Scrum es el reconocimiento de que durante un proyecto los clientes pueden cambiar de idea sobre lo que quieren y necesitan, y que los desafíos impredecibles no pueden ser fácilmente enfrentados de una forma predictiva y planificada. Por lo tanto, Scrum adopta una aproximación pragmática, aceptando que el problema no puede ser completamente entendido o definido, y centrándose en maximizar la capacidad del equipo de entregar rápidamente y responder a requisitos emergentes.

Existen varias implementaciones de sistemas para gestionar el proceso de Scrum, que van desde post-it y pizarras, hasta paquetes de software. Una de las mayores ventajas de Scrum es que es muy fácil de aprender, y requiere muy poco esfuerzo para comenzarse a utilizar. En nuestro caso, se ha optado por el uso de post-it de diferentes colores según las prioridades para cada tarea (naranja-alta, amarilla-media, verda-baja) y chinchetas de colores para diferenciar las asignaciones de cada tarea a cada miembro del equipo. Un ejemplo de nuestra aplicación de la metodología se puede observar en la figura \ref{fig:scrum}, en la que se puede comprobar la pizarra, post-it y chinchetas que utilizamos.  

%figura metodologia scrum
\begin{figure}[t]
\begin{center}
\includegraphics[scale=0.08]{includes/images/scrum.jpg}
\caption{Aplicación de la metodología Scrum}
\label{fig:scrum}
\end{center}
\end{figure}

\section{Reuniones en Scrum}
\label{capitulo5:reuniones}
\textbf{Daily Scrum}
Cada día de un sprint, se realiza la reunión sobre el estado de un proyecto. El Scrum tiene unas guías específicas:
La reunión comienza puntualmente a su hora. A menudo hay castigos (acordados por el equipo) para quien llegue tarde (por ejemplo: dinero, flexiones, llevar colgando una gallina de plástico del cuello, etc.).
La reunión tiene una duración fija de 15 minutos, independientemente del tamaño del equipo.
Todos los asistentes deben mantenerse de pie (esto ayuda a mantener la reunión corta).
La reunión debe ocurrir en la misma ubicación y a la misma hora todos los días.
Durante la reunión, cada miembro del equipo contesta a tres preguntas:
¿Qué has hecho desde ayer?
¿Qué es lo que estás planeando hacer hoy?
¿Has tenido algún problema que te haya impedido alcanzar tu objetivo?

\textbf{Reunión de Planificación del Sprint (Sprint Planning Meeting)}
Al inicio del ciclo Sprint (cada 15 o 30 días), una Reunión de Planificación del Sprint se lleva a cabo.
Se selecciona qué trabajo se hará, 
se preparara, con el equipo completo, el Sprint Backlog que detalla el tiempo que llevará hacer el trabajo.
Se identificará y comunicará qué parte del trabajo es probable que se realice durante el actual Sprint.
Como máximo durará 8 horas.
Al final del ciclo Sprint, se llevaran a cabo dos reuniones: la Reunión de Revisión del Sprint y la Retrospectiva del Sprint.

\textbf{Reunión de Revisión del Sprint (Sprint Review Meeting)}
Se Revisa el trabajo que fue completado y no completado en 4 horas como máximo.

\textbf{Retrospectiva del Sprint (Sprint Retrospective)}
Después de cada sprint, se lleva a cabo una retrospectiva del sprint, en la cual todos los miembros del equipo dejan sus impresiones sobre el sprint recién terminado. El propósito de la retrospectiva es realizar una mejora continua del proceso. Esta reunión tiene un tiempo fijo de cuatro horas.
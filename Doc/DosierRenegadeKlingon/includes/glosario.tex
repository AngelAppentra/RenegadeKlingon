%\input miscomandos.tex % Comandos definidos por el autor

%\documentclass[a4paper,12pt]{book} 

% Incluir los paquetes necesarios 
%\usepackage[latin1]{inputenc} % Caracteres con acentos. 
%\usepackage[spanish]{babel}
%\usepackage{latexsym} % Smbolos 
%\usepackage[pdftex=true,colorlinks=true,plainpages=false]{hyperref} % Soporte hipertexto
%\usepackage[pdftex]{graphicx} %Inclusin de grficos PDFLaTeX
%\usepackage{color}
%\DeclareGraphicsExtensions{.png,.pdf,.jpg}
%\sloppy % suaviza las reglas de ruptura de lneas de LaTeX
%\renewcommand{\baselinestretch}{1.5} %espacio entre lineas

% Ttulo, autor, fecha. 
\title{glosario} 
\author{Angel Baltar Díaz}
\date{\Large Enero , 2013} 

%\begin{document} % Inicio del documento

\chapter{Glosario de Términos}
\label{glosario}

\begin{itemize}

\item \textbf{R-type} Saga de juegos arcade de naves espaciales, género matamarcianos, que comienza en 1987 y saca multitud de títulos como R-Type, R-TypeII, R-Type Leo...

\item \textbf{parallax} Sistema de movimiento de objetos en el que el fondo se mueve a menor velocidad que los objetos en planos más cercanos creando una cierta sensacion de 3D sin realmente existir 3D en si mismo.

\item \textbf{tile} Unidad básica en el diseño de mapas, es una celda del mapa, el objeto más pequeño que puede cargarse es como mucho de tamaño 1x1 tile.

\end{itemize}


%\end{document}
